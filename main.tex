\documentclass[a4paper, 11pt]{book}  %  
%\documentclass[a4paper, 11pt]{article}  %  
%\documentclass[a4paper, 11pt]{report}  %  
 
 
\usepackage[brazilian]{babel}  % linguaguem  acentuação
\usepackage[utf8]{inputenc}  %codificacao de entrada
\usepackage[T1]{fontenc}   %codificacao de saída
\usepackage{verbatim} % para que os comandos sejam 'vistos' como texto
\usepackage{color} % para texto colorido
% Customização da página/paginação
\usepackage{float} % para uso do H (maiúsculo) no table
\usepackage{multirow} % para mesclar linhas
 
\usepackage{graphicx}
\usepackage[left=1cm,top=2cm,right=1cm,bottom=1cm]{geometry}
 
% pacotes para figuras ou subfiguras
\usepackage{subfigure}
\usepackage{caption}
\usepackage{subcaption}
\usepackage{subfloat}
 
% Inserir este pacote antes do abntex2cite
% devido a dependências (conflitos)
\usepackage{hyperref}
\hypersetup{
    colorlinks=true,
    citecolor=blue,
    %linkcolor=blue,
    %filecolor=magenta,    
    urlcolor=cyan,
    %pdftitle={Overleaf Example},    %pdfpagemode=FullScreen
    }
 
 
%referências
\usepackage[alf, abnt-emphasize=bf]{abntex2cite}
 
 
\usepackage{multicol}
\usepackage{lipsum}
% para gerar texto 
% sem sentido/aleatório
 
 
\author{Daniel Barreto de Oliveira}
\title{Introdução ao LaTeX}
\date{\today}
%\date{\nodate}
\begin{document}
\maketitle
\chapter{Primeiro Capítulo}
%chapter*{Primeiro Capitulo}
%Não aparece a expressão
%Capítulo 1
\section{Meu primeiro documento Latex}
\subsection{Escrevendo um texto simples}
Para Digitar um texto comum, basta simplesmente escrever.
Texto com uma caixa: \fbox{média e variancia}

Caracteres especiais como porcentagem 12\% e cifrão R\$1236,00 são escritos assim

A temperatura assume um valor: \{teste\}


\subsection{Escrevendo fórmulas matemáticas}
Exemplo de fórmulas matemáticas: \newline
Exemplo de fórmulas matemáticas: \\
Teste
Teste
Escrevendo formulas na linha: Seja $f$ definida por $f(x)=x+b$

Uma formula em uma única linha (só a fórmula na linha).
$$\overline{x} = \frac{\sum x_{i}}{n}$$
$$\bar{x} = \frac{\sum x_{i}}{n}$$

Outra forma de escrever caracteres especiais:
\begin{verbatim}
    \displaystyle \\ %% && $ % { e }
\end{verbatim}

Mais um ambiente matemático

\[
    \bar{x}_{2} =   \frac{\sum{x_{i}}}{n}
\]

Um ambiente que permite enumerar as fórmulas:

\begin{equation}
    \bar{x}_{aa} = \frac{\sum{x_{i}}}{n}
    \label{mediaaa}
\end{equation}

\begin{equation}
    \bar{x}_{ab} = \frac{\sum_{i=1}^{n}{x_{i}}}{n}
    %Sem label não é possível retomar a equação
\end{equation}

\begin{equation}
    \bar{x}_{ac} = \frac{\displaystyle \sum_{i=1}^{n}{x_{i}}}{n}
\end{equation}

%O \nonumber não funciona no equation

\begin{equation}
    \bar{x}_{ad} = \frac{\displaystyle \sum_{i=1}^{n}{x_{i}}}{n} \nonumber
\end{equation}

Teste de escrita



Referência \ref{mediaaa}
\textit{equation}



\end{document}


