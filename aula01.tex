\documentclass[a4paper, 11pt]{book}  %  
%\documentclass[a4paper, 11pt]{article}  %  
%\documentclass[a4paper, 11pt]{report}  %  
 
 
\usepackage[brazilian]{babel}  % linguaguem  acentuação
\usepackage[utf8]{inputenc}  %codificacao de entrada
\usepackage[T1]{fontenc}   %codificacao de saída
\usepackage{verbatim} % para que os comandos sejam 'vistos' como texto
\usepackage{color} % para texto colorido
% Customização da página/paginação
\usepackage{float} % para uso do H (maiúsculo) no table
\usepackage{multirow} % para mesclar linhas
 
\usepackage{graphicx}
\usepackage[left=1cm,top=2cm,right=1cm,bottom=1cm]{geometry}
 
% pacotes para figuras ou subfiguras
\usepackage{subfigure}
\usepackage{caption}
\usepackage{subcaption}
\usepackage{subfloat}
 
% Inserir este pacote antes do abntex2cite
% devido a dependências (conflitos)
\usepackage{hyperref}
\hypersetup{
    colorlinks=true,
    citecolor=blue,
    %linkcolor=blue,
    %filecolor=magenta,    
    urlcolor=cyan,
    %pdftitle={Overleaf Example},    %pdfpagemode=FullScreen
    }
 
 
%referências
\usepackage[alf, abnt-emphasize=bf]{abntex2cite}
 
 
\usepackage{multicol}
\usepackage{lipsum}
% para gerar texto 
% sem sentido/aleatório
 
 
\author{Daniel Barreto de Oliveira}
\title{Introdução ao LaTeX}
\date{\today}
%\date{\nodate}
\begin{document}
\maketitle
\chapter{Primeiro Capítulo} %chapter*{Primeiro Capitulo}
\section{Meu primeiro documento Latex}
\subsection{Escrevendo um texto simples}
Para Digitar um texto comum, basta simplesmente escrever.
Texto com uma caixa: \fbox{média e variancia}

Caracteres especiais como porcentagem 12\% e cifrão R\$1236,00 são escritos assim

A temperatura assume um valor: \{teste\}


\subsection{Escrevendo fórmulas matemáticas}
Exemplo de fórmulas matemáticas: \newline
Exemplo de fórmulas matemáticas: \\
Teste
Teste
Escrevendo formulas na linha: Seja $f$ definida por $f(x)=x+b$

Uma formula em uma única linha (só a fórmula na linha).
$$\overline{x} = \frac{\sum x_{i}}{n}$$
$$\bar{x} = \frac{\sum x_{i}}{n}$$

Outra forma de escrever caracteres especiais:
\begin{verbatim}
    \displaystyle \\ %% && $ % { e }
\end{verbatim}

Mais um ambiente matemático

\[
    \bar{x}_{2} =   \frac{\sum{x_{i}}}{n}
\]

Um ambiente que permite enumerar as fórmulas:

\begin{equation}
    \bar{x}_{aa} = \frac{\sum{x_{i}}}{n}
    \label{mediaaa}
\end{equation}

\begin{equation}
    \bar{x}_{ab} = \frac{\sum_{i=1}^{n}{x_{i}}}{n}
    %Sem label não é possível retomar a equação
\end{equation}

\begin{equation}
    \bar{x}_{ac} = \frac{\displaystyle \sum_{i=1}^{n}{x_{i}}}{n}
    \label{media4}
\end{equation}

%O \nonumber não funciona no equation

\begin{equation}
    \bar{x}_{ad} = \frac{\displaystyle \sum_{i=1}^{n}{x_{i}}}{n} \nonumber
\end{equation}

%Não numerada
%\label não funciona com \nonumber
\begin{eqnarray}
    \bar{x}_{ad} = \frac{\displaystyle \sum_{i=1}^{n}{x_{i}}}{n} \nonumber
\end{eqnarray}

\begin{eqnarray}
    \bar{x}_{4} = \frac{\sum_{i=1}^{n}{x_{i}}}{n} \nonumber \\ 
    S^2 = \frac{\sum_{i=1}^{n}{(x_{i} - \bar{x})^2}}{n-1} \nonumber
\end{eqnarray}

\begin{eqnarray}
    \bar{x}_{4} &=& \frac{\sum_{i=1}^{n}{x_{i}}}{n} \nonumber \\ 
    S^2 &=& \frac{\sum_{i=1}^{n}{(x_{i} - \bar{x})^2}}{n-1} \nonumber
\end{eqnarray}

\begin{eqnarray}
    \bar{x}_{4} = \frac{\sum_{i=1}^{n}{x_{i}}}{n}
    \quad       % Tambem existe o \.
    S^2 = \frac{\sum_{i=1}^{n}{(x_{i} - \bar{x})^2}}{n-1} \nonumber
\end{eqnarray}

%\quada , \hspace espaçamento horizontal
%\textrm para inserir texto no ambiente matemático

\begin{eqnarray}
    \bar{x}_{4} = \frac{\sum_{i=1}^{n}{x_{i}}}{n}
    \quad \textrm{ e }\hspace{1cm}
    \quad S^2 = \frac{\sum_{i=1}^{n}{(x_{i} - \bar{x})^2}}{n-1} \nonumber
    \label{media4}  % Teste para refência
\end{eqnarray}
    
\begin{eqnarray}
    \bar{x}_{5} = \frac{\displaystyle \sum_{i=1}^{n}{x_{i}}}{n}
    \label{media5}
\end{eqnarray}

Referência \ref{mediaaa}
\textit{equation}
Teste de texto \textit{itálico} com comando
\begin{verbatim}
    ctrl + i
\end{verbatim}

Ao referenciar a média 4 é mostrado a média mais próxima
\ref{media4}

\vspace{5cm}
Uso do vspace
\vspace{5cm}

Exemplo de notação para estimador: $\hat{\theta}$ em que $\theta$ é o parâmetro, definido no espaço paramétrico $\Theta$

Exemplo de integral
\begin{eqnarray}
    \int_{-\infty}^{\infty}\frac{1}{\sqrt{2 \times \pi \times \sigma^2}}
    exp\{ -\frac{(x-\mu)^2}{2 \times \sigma^2}\} dx \nonumber
\end{eqnarray}
Arrumando a expressão:
\begin{eqnarray}
    \int_{-\infty}^{\infty}
    \frac{1}{\sqrt{2 \pi \sigma^2}}
    exp\left\{ -\frac{(x-\mu)^2}{2 \sigma^2}\right\} dx
    \label{int_normal}
\end{eqnarray}

\subsection{Escrevendo um texto em itens e exemplificando mudança no tamanho da fonte}

Os parâmetros mais conhecidos:

\begin{itemize}
    \item media, $\mu$;
    \item {\Large media, $\mu$;}
    \item {\LARGE variância, $\sigma^2$,}
    \item desvio padrão, $\sigma$
\end{itemize}

\begin{itemize}
    \item [i -] media, $\mu$;
    \item [ii -] {\Large media, $\mu$;}
    \item [iii -] {\LARGE variância, $\sigma^2$;}
    \item [iv -] desvio padrão, $\sigma$
\end{itemize}

\begin{enumerate}
    \item {media}
    \item {variância}
\end{enumerate}

%Nova página
\newpage
\subsection{Como alinhar o texto à direita, esquerda e centralizar}

\begin{center}
    Este texto está centralizado.
\end{center}

\begin{flushleft}
   Este texto está alinhado à esquerda.
\end{flushleft}

\begin{flushright}
    Este texto está alinhado à esquerda.
\end{flushright}
    
\subsection{Texto colorido}

%textbf
\begin{verbatim}
    ctrl + b 
\end{verbatim}

{\color{black} preto ou \textbf{negrito}} \\
{\color{blue} azul ou \textbf{azul}} \\


\subsection{Alguns exemplos de tabela}

%Utilize um gerador de tabela online para LaTeX
%\begin{table}[htbp] %Aqui, ou no topo, ou no bottom, ou no topo da próxima página
\begin{table}[H]    %H tabela aqui, paco à parte (está no pré-ambulo)
    \centering
    \caption{Exemplo de uma tabela de uma única entrada}
    \begin{tabular}{c|c}
         Curso& Qtd. Alunos  \\
         Eng. Alétrica& 100 \\ 
    \end{tabular}
    \label{tab:my_label}
\end{table}

\begin{table}[H]
\centering
\caption{Exemplo de uma tabela feita por um gerador de tabela LaTeX}
\label{tab:curso}
\begin{tabular}{lc}
\hline
\hline
\textbf{Curso} & \textbf{Quantidade de Aluno} \\ \hline
Eng. Elétrica  & 100                          \\
Estatística    & 50                           \\
Matemática     & 50                           \\
Medicina       & 150                          \\
Odontologia    & 120                          \\
Zootecnia      & 110                          \\ \hline
\hline
\end{tabular}
\end{table}

% Please add the following required packages to your document preamble:
% \usepackage{multirow}
\begin{table}[H]
\centering
\caption{Exemplo de tabela de duas variáveis entrada.}
\label{tab:duas_entrdas}
\begin{tabular}{lc|c}
\hline
\multirow{2}{*}{Curso} & \multicolumn{2}{l}{Forma de entrada} \\ \cline{2-3} 
                       & Enem             & Outra             \\ \hline
A                      & 3                & 2                 \\
B                      & 2                & 1                 \\
C                      & 3                & 2                 \\ \hline            
\end{tabular}
\end{table}

Na tabela \ref{tab:curso}   %\ref* não deixa destacado

\subsection{Alguns exemplos de figuras}

\begin{figure}[H]
    \centering
    \includegraphics{figuras/normal-padrao.png}
    \caption{Distribuição Normal Padrão}
    \label{fig:norma_padrão}
\end{figure}

\begin{figure}[H]
    \centering
    \includegraphics[scale=0.5]{figuras/normal-padrao.png}
    \caption{Distribuição Normal Padrão}
    \label{fig:norma_padrão_reduzida}
\end{figure}

\begin{figure}[H]
    \centering
    \includegraphics[width=5cm, height=10cm]{figuras/normal-padrao.png}
    \caption{Distribuição Normal Padrão}
    \label{fig:normal-padrao-lagura-altura}
\end{figure}

Figuras \ref{fig:norma_padrão}, \ref{fig:norma_padrão_reduzida}, \ref{fig:normal-padrao-lagura-altura}

\end{document}


